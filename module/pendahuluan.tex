\chapter{Pendahuluan}

\section{Latar Belakang}

Di era transformasi digital saat ini, data telah menjadi aset strategis bagi organisasi di berbagai sektor. Volume data yang terus meningkat, baik dari transaksi internal, interaksi pelanggan, hingga media sosial, menciptakan peluang besar sekaligus tantangan dalam pengambilan keputusan yang efektif dan berbasis bukti. Istilah \textit{big data} mencerminkan kompleksitas data modern yang ditandai oleh karakteristik Volume, Variety, Velocity, Veracity, dan Value (5V).

Bagi kalangan manajerial dan pengambil kebijakan, pemahaman mengenai big data tidak lagi terbatas pada aspek teknis, melainkan lebih pada bagaimana data dapat dimanfaatkan untuk menciptakan nilai bisnis, meningkatkan efisiensi operasional, memahami perilaku konsumen, serta mendukung inovasi berbasis analitik.

\section{Tujuan Pembelajaran}

Mata kuliah ini bertujuan untuk membekali mahasiswa pascasarjana dari latar belakang manajemen dan bisnis dengan pemahaman konseptual dan keterampilan praktis dalam mengelola dan memanfaatkan big data. Tanpa memerlukan latar belakang pemrograman atau teknologi informasi, mahasiswa akan diperkenalkan pada teknologi big data, pendekatan analitik, serta kerangka pengambilan keputusan berbasis data.

Secara khusus, tujuan pembelajaran mencakup:

\begin{itemize}
	\item Memahami konsep, karakteristik, dan peran strategis big data dalam organisasi.
	\item Mampu menjelaskan infrastruktur dan arsitektur teknologi big data secara konseptual.
	\item Menggunakan alat bantu visual (Power BI, Orange, KNIME) untuk eksplorasi data, analisis prediktif, dan visualisasi.
	\item Menganalisis kasus penggunaan big data dalam berbagai sektor industri.
	\item Mengidentifikasi tantangan etika, tata kelola, dan monetisasi data dalam konteks organisasi.
\end{itemize}

\section{Ruang Lingkup Materi}

Materi dalam mata kuliah ini disusun secara bertahap, mencakup empat area pembelajaran utama:

\begin{enumerate}
	\item \textbf{Dasar-dasar dan Strategi Big Data:} pengenalan konsep big data, kerangka kerja strategis seperti \textit{Big Data Value Chain}, \textit{5Vs}, \textit{Big Data Maturity Model} (BDMM), dan arsitektur teknologi data.
	
	\item \textbf{Pengelolaan dan Pemrosesan Data:} mencakup pemahaman basis data (SQL dan NoSQL), proses pembersihan data, dan integrasi sumber data.
	
	\item \textbf{Analisis dan Pengambilan Keputusan:} penggunaan BI dan machine learning untuk segmentasi pelanggan, prediksi perilaku, dan analisis sentimen pelanggan.
	
	\item \textbf{Manfaat, Etika, dan Tata Kelola:} penilaian nilai ekonomi dari data (monetisasi), kerangka kerja pengambilan keputusan berbasis data (CRISP-DM, OSEMN), serta isu etika dan tata kelola data.
\end{enumerate}


\begin{longtable}{|p{0.03\textwidth}|p{0.2\textwidth}|p{0.3\textwidth}|p{0.35\textwidth}|}
	\caption{Course Plan: Big Data for Business (Postgraduate, Management Focus)}\label{tab:course_plan} \\
	\hline
	\textbf{No} & \textbf{Topik} & \textbf{Deskripsi} & \textbf{Frameworks / Tools} \\
	\hline
	\endfirsthead
	
	\multicolumn{4}{c}%
	{{\tablename\ \thetable{} -- lanjutan dari halaman sebelumnya}} \\
	\hline
	\textbf{No} & \textbf{Topik} & \textbf{Deskripsi} & \textbf{Frameworks / Tools} \\
	\hline
	\endhead
	
	\hline \multicolumn{4}{r}{{Bersambung ke halaman berikutnya}} \\
	\endfoot
	
	\hline
	\endlastfoot
	
	1 & Introduction to Big Data in Business & Definitions, business impact, why big data matters & Case study. \\
	\hline
	2 & Big Data Strategy \& Frameworks Overview & Introduce 5Vs, Big Data Value Chain, BDMM, DAMA & -- \\
	\hline
	3 & Big Data Architecture \& Value Chain & End-to-end process: from data sources, ingestion, storage, processing, to BI and analytics presentation & Big Data Value Chain (Capture–Process–Analyse–Visualise–Decide), BDMM, DAMA. Architecture mapping activity. \\
	\hline
	4 & Understanding Databases \& SQL & Structured data and basic querying & Big Data Value Chain (Capture), DAMA (Architecture). SQL tools (DB Fiddle). \\
	\hline
	5 & NoSQL \& Semi-Structured Data & JSON, MongoDB, flexibility in schema design & Big Data Value Chain (Capture), 5Vs (Variety). MongoDB Atlas. \\
	\hline
	6 & Big Data Processing Technologies & Distributed processing frameworks: Hadoop, MapReduce, Spark, Kafka, stream vs batch concepts & Big Data Value Chain (Process–Analyse). Demos of Spark, Hadoop HDFS overview, Kafka stream processing concept. \\
	\hline
	7 & Data Cleaning \& Preparation & Data quality, integration, consistency & Big Data Value Chain (Curate), DAMA (Quality), 5Vs (Veracity). KNIME workflow. \\
	\hline
	8 & Business Intelligence \& Dashboards & KPIs, dashboards, visualisation & Big Data Value Chain (Analyse–Visualise), DAMA. Power BI. \\
	\hline
	9 & Introduction to Machine Learning & ML vs BI, supervised learning & Big Data Value Chain (Analyse), BDMM (Analytics readiness). Orange ML demo. \\
	\hline
	10 & Customer Segmentation (Clustering) & Market segmentation with unsupervised learning & Big Data Value Chain (Analyse), DAMA (Mining), 5Vs (Variety). Orange: k-means. \\
	\hline
	11 & Prediction \& Forecasting & Churn, revenue, or risk prediction & Big Data Value Chain (Analyse–Decide), BDMM. Orange: regression/classification. \\
	\hline
	12 & Text \& Sentiment Analysis & Review/opinion analysis with NLP & Big Data Value Chain (Capture–Analyse), 5Vs (Veracity). Orange: sentiment analysis. \\
	\hline
	13 & Data Monetisation \& Value Realisation & Economic/business value from data & Big Data Value Chain (full chain), BDMM (Value). Value creation strategy. \\
	\hline
	14 & Ethics, Governance \& Maturity Review & Data privacy, governance, ethical use of data & DAMA (Governance), BDMM (Audit). BDMM self-assessment. \\
	\hline
	
\end{longtable}




\section{Metodologi Pembelajaran}

Pembelajaran dilakukan melalui pendekatan interaktif yang menggabungkan ceramah konseptual, studi kasus industri, demonstrasi alat bantu visual, dan latihan praktik berbasis data nyata. Mahasiswa akan bekerja secara individu maupun berkelompok dalam menganalisis permasalahan bisnis dengan pendekatan berbasis data.



\section{Manfaat yang Diharapkan}

Setelah mengikuti mata kuliah ini, mahasiswa diharapkan mampu mengambil peran strategis dalam perencanaan, pengelolaan, dan pemanfaatan data besar untuk mendukung tujuan organisasi. Selain itu, mahasiswa juga memiliki dasar yang kuat untuk memahami dan mengevaluasi proyek transformasi digital berbasis data.


\section{Materi Perkuliahan}

Tabel~\ref{tab:course_plan} menjelaskan struktur 14 pertemuan untuk mata kuliah \textit{Big Data for Business} yang dirancang khusus bagi mahasiswa pascasarjana dari latar belakang manajemen atau non-TIK. Setiap sesi disusun secara bertahap untuk membekali mahasiswa dengan pemahaman praktis mengenai konsep, teknologi, dan penerapan big data dalam konteks organisasi. Materi perkuliahan dibagi ke dalam empat tahapan pembelajaran utama:

\begin{enumerate}
	\item \textbf{Dasar-dasar, Strategi, dan Arsitektur Big Data (Sesi 1--3)} \\
	Mahasiswa diperkenalkan pada konsep dasar big data, kerangka kerja strategis seperti \textit{Big Data Value Chain}, \textit{5Vs}, \textit{BDMM}, serta arsitektur big data end-to-end dari sumber data hingga analitik bisnis.
	
	\item \textbf{Teknologi Pengelolaan dan Pemrosesan Data (Sesi 4--7)} \\
	Fokus pada pemahaman basis data relasional (SQL), data semi-terstruktur (NoSQL), serta teknologi pemrosesan big data seperti Hadoop, MapReduce, Spark, dan Kafka. Termasuk juga pembersihan dan integrasi data menggunakan alat bantu visual seperti KNIME.
	
	\item \textbf{Analitik Bisnis dan Pembelajaran Mesin (Sesi 8--12)} \\
	Mahasiswa mengeksplorasi penerapan \textit{Business Intelligence} menggunakan Power BI, serta pembelajaran mesin (machine learning) menggunakan Orange untuk segmentasi pelanggan, prediksi, dan analisis sentimen.
	
	\item \textbf{Manfaat, Tata Kelola, dan Etika Big Data (Sesi 13--14)} \\
	Tahap akhir mengkaji strategi monetisasi dan realisasi nilai data, serta isu etika, tata kelola data, dan evaluasi kematangan big data melalui pendekatan DAMA dan BDMM.
\end{enumerate}

Kegiatan pembelajaran memanfaatkan pendekatan praktis berbasis alat bantu visual tanpa pemrograman, serta studi kasus dari industri untuk menghubungkan teori dan praktik. Hasil akhir yang diharapkan adalah pemahaman menyeluruh tentang bagaimana data besar mendukung keputusan dan transformasi bisnis dalam berbagai sektor industri.
