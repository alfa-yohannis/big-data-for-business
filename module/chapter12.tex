\chapter{Model Kematangan BI dan AI TDWI}
\label{chap:tdwi_bi_ai_maturity}

\section{Nilai dan Tujuan Model}

Model kematangan Business Intelligence (BI) dan Artificial Intelligence (AI) yang dikembangkan oleh 
\textit{Transforming Data with Intelligence} (TDWI) berfungsi sebagai kerangka kerja untuk menilai sejauh mana organisasi 
telah memanfaatkan data, analitik, dan teknologi AI secara strategis.  

Tujuan utama model ini adalah:
\begin{itemize}
	\item \textbf{Mengukur posisi saat ini:} membantu organisasi memahami di mana posisi mereka pada spektrum kematangan analitik.
	\item \textbf{Mengidentifikasi kesenjangan:} menemukan area kelemahan seperti infrastruktur data, tata kelola, atau keterampilan SDM.
	\item \textbf{Mengarahkan transformasi:} memberikan panduan peningkatan bertahap agar BI dan AI selaras dengan strategi bisnis.
	\item \textbf{Mengoptimalkan nilai bisnis:} memastikan investasi data dan AI menghasilkan dampak nyata pada efisiensi, inovasi, dan pertumbuhan \cite{tdwi2025, eckerson2009tdwi}.
\end{itemize}

Menurut TDWI, organisasi yang matang dalam BI dan AI akan mampu menciptakan budaya \textit{data-driven}, 
meningkatkan kecepatan pengambilan keputusan, dan mengelola risiko implementasi AI secara lebih efektif. 
Selain itu, model ini juga memandu integrasi teknologi terkini seperti \textit{cloud data platform}, \textit{MLOps}, dan \textit{GenAI} 
ke dalam strategi analitik modern \cite{fernandez2023ai_maturity}.

\section{Pengantar}

Perkembangan pesat dalam bidang \textit{Business Intelligence} (BI) dan \textit{Artificial Intelligence} (AI) telah 
mengubah cara organisasi memanfaatkan data untuk mendukung pengambilan keputusan strategis. BI berfokus pada 
penyediaan wawasan melalui pelaporan, analitik deskriptif, dan visualisasi data, sementara AI memperluas kapabilitas 
ini dengan menghadirkan kemampuan prediksi, rekomendasi, dan otomatisasi berbasis model pembelajaran mesin 
\cite{chen2012, davenport2018}. 

Dalam praktiknya, transformasi digital menuntut organisasi untuk mengintegrasikan BI dan AI secara terpadu. 
Pendekatan ini tidak hanya meningkatkan efisiensi operasional, tetapi juga membuka peluang inovasi baru, 
misalnya melalui personalisasi layanan pelanggan, deteksi anomali dalam rantai pasok, serta penerapan \textit{Generative AI} 
untuk meningkatkan produktivitas pengetahuan \cite{fernandez2020, lim2023adoption}.  

Namun, tingkat adopsi dan efektivitas BI dan AI sangat bervariasi antar organisasi. Banyak organisasi yang masih berada 
pada tahap awal, dengan pemanfaatan BI sebatas laporan tradisional dan AI hanya sebatas eksperimen terbatas. 
Sebaliknya, organisasi yang sudah matang biasanya memiliki arsitektur data yang terintegrasi, praktik tata kelola yang 
kuat, serta budaya \textit{data-driven} yang tertanam pada seluruh level manajemen \cite{liao2021business, mukherjee2022comparative}.  

Untuk membantu organisasi memahami posisi mereka serta menentukan langkah strategis selanjutnya, 
\textit{The Data Warehousing Institute} (TDWI) memperkenalkan \textbf{Model Kematangan BI dan AI}. 
Model ini berfungsi sebagai peta jalan (roadmap) bagi organisasi untuk mengukur kapabilitas data dan analitik mereka, 
mengidentifikasi kesenjangan, serta menyusun prioritas peningkatan yang selaras dengan tujuan bisnis jangka panjang.
\section{Tren Kunci BI dan AI}

Perkembangan BI dan AI dewasa ini menunjukkan sejumlah tren utama yang memengaruhi arah strategi data dan analitik 
organisasi. Tren-tren ini bukan hanya bersifat teknologi, tetapi juga menyangkut aspek tata kelola, budaya, dan 
perubahan peran sumber daya manusia.

\subsection*{1. Self-Service Analytics dan Generative AI}
Tren pertama adalah adopsi \textit{self-service analytics} yang semakin diperkaya oleh kemampuan 
\textit{Generative AI} (GenAI). Pengguna bisnis kini dapat mengajukan pertanyaan langsung dalam bahasa alami 
dan mendapatkan jawaban analitis dalam bentuk narasi, visualisasi, atau bahkan rekomendasi aksi. Hal ini 
meningkatkan demokratisasi data, meskipun tetap menimbulkan tantangan terkait kualitas data dan kontrol tata kelola 
\cite{imhoff2014ssbi, chen2021aibi}.  

\subsection*{2. Tata Kelola Data dan AI}
Seiring meningkatnya adopsi AI, tata kelola (governance) menjadi semakin penting. Tidak hanya mencakup 
data governance, tetapi juga \textit{AI governance} yang mengatur transparansi model, 
pengendalian bias, serta kepatuhan terhadap regulasi seperti GDPR. Organisasi dituntut menyeimbangkan 
inovasi dengan kepercayaan dan akuntabilitas \cite{khatri2010data, zwitter2014}.  

\subsection*{3. Integrasi Infrastruktur Cloud dan Data Lakehouse}
Platform berbasis cloud, termasuk arsitektur \textit{data lakehouse}, semakin dominan dalam mendukung 
analitik skala besar. Integrasi ini memungkinkan organisasi menggabungkan fleksibilitas data lake dengan 
konsistensi data warehouse, sehingga mendukung kebutuhan BI tradisional sekaligus eksplorasi AI 
\cite{armbrust2021lakehouse}.  

\subsection*{4. Peran Baru: MLOps dan AIOps}
Untuk memastikan model AI berjalan secara berkelanjutan, peran baru seperti MLOps dan AIOps mulai 
menjadi kebutuhan. MLOps menekankan siklus hidup penuh model (pengembangan, deployment, monitoring), 
sedangkan AIOps berfokus pada pemanfaatan AI untuk mengoptimalkan operasi TI. Kedua peran ini 
memperkuat kemampuan organisasi dalam mengoperasionalkan AI pada skala besar \cite{goyal2022}.  

\subsection*{5. Data sebagai Produk dan Monetisasi}
Tren lain adalah pergeseran pandangan bahwa data bukan hanya aset internal, tetapi juga dapat 
dimonetisasi sebagai produk. Konsep ini menuntut organisasi untuk mengelola data dengan prinsip 
\textit{product management}, mulai dari kualitas, dokumentasi, hingga siklus hidup nilai data 
\cite{labreuche2020}.  

Secara keseluruhan, tren-tren ini menegaskan bahwa integrasi BI dan AI bukan lagi sekadar 
persoalan teknologi, melainkan transformasi menyeluruh yang melibatkan strategi bisnis, 
budaya organisasi, dan tata kelola yang kuat.

\section{Dimensi Model}



\begin{figure}[h]
	\centering
	\begin{tikzpicture}[font=\small, node distance=1.2cm]
		
		% style node umum
		\tikzset{
			dim/.style={draw=black!40, rounded corners=2pt, text width=3cm, minimum height=1cm, align=center, font=\bfseries},
			levels/.style={draw=black!30, text width=11cm, minimum height=1cm, align=center}
		}
		
		% 5 node dimensi secara vertikal + warna pastel berbeda
		\node[dim, fill=red!10] (d1) {Data dan Infrastruktur};
		\node[dim, fill=blue!10, below of=d1] (d2) {Analitik dan Teknologi};
		\node[dim, fill=green!10, below of=d2] (d3) {Tata Kelola dan Manajemen};
		\node[dim, fill=orange!15, below of=d3] (d4) {Organisasi dan Budaya};
		\node[dim, fill=purple!10, below of=d4] (d5) {Strategi dan Nilai Bisnis};
		
		% 5 node level di kanan masing-masing dimensi, dengan warna senada
		\node[levels, fill=red!5, right=0.2cm of d1] {Nascent $\rightarrow$ Early $\rightarrow$ Established $\rightarrow$ Mature $\rightarrow$ Advanced/Visionary};
		\node[levels, fill=blue!5, right=0.2cm of d2] {Nascent $\rightarrow$ Early $\rightarrow$ Established $\rightarrow$ Mature $\rightarrow$ Advanced/Visionary};
		\node[levels, fill=green!5, right=0.2cm of d3] {Nascent $\rightarrow$ Early $\rightarrow$ Established $\rightarrow$ Mature $\rightarrow$ Advanced/Visionary};
		\node[levels, fill=orange!8, right=0.2cm of d4] {Nascent $\rightarrow$ Early $\rightarrow$ Established $\rightarrow$ Mature $\rightarrow$ Advanced/Visionary};
		\node[levels, fill=purple!5, right=0.2cm of d5] {Nascent $\rightarrow$ Early $\rightarrow$ Established $\rightarrow$ Mature $\rightarrow$ Advanced/Visionary};
		
	\end{tikzpicture}
	\caption{Lima dimensi model kematangan BI \& AI dengan warna pastel berbeda untuk tiap baris.}
	\label{fig:dimensi-vertical}
\end{figure}





Model kematangan BI dan AI yang dikembangkan TDWI disusun berdasarkan sejumlah dimensi inti yang 
mencerminkan aspek-aspek penting dalam pengelolaan data dan analitik (Gambar~\ref{fig:dimensi-vertical}). Setiap dimensi berfungsi 
sebagai lensa evaluasi untuk memahami kekuatan dan kelemahan organisasi dalam perjalanan menuju 
organisasi berbasis data yang matang.  

Secara umum, terdapat lima dimensi utama yang menjadi pilar dalam model ini:

\subsection*{1. Data dan Infrastruktur}
Dimensi ini berfokus pada kualitas, ketersediaan, dan manajemen data. Organisasi pada tahap awal 
biasanya masih menggunakan data yang terfragmentasi dan silo, sementara organisasi yang matang 
telah mengintegrasikan \textit{data warehouse}, \textit{data lake}, maupun \textit{lakehouse} 
dengan tata kelola yang kuat. Faktor seperti integrasi data real-time, metadata management, 
dan kualitas data menjadi indikator utama \cite{kimball2013, armbrust2020delta}.

\subsection*{2. Analitik dan Teknologi}
Dimensi ini mencerminkan tingkat kecanggihan analitik yang digunakan, mulai dari laporan deskriptif 
hingga analitik prediktif, preskriptif, dan penerapan AI. Organisasi dengan kematangan tinggi 
mampu mengoperasionalkan model pembelajaran mesin melalui MLOps, serta memanfaatkan teknologi 
baru seperti \textit{Generative AI} untuk memperluas kapabilitas analitik \cite{davenport2018, chen2021aibi}.

\subsection*{3. Tata Kelola dan Manajemen}
Governance mencakup kebijakan, standar, serta mekanisme pengendalian dalam mengelola data dan AI. 
Hal ini termasuk kepatuhan regulasi, manajemen risiko, dan etika penggunaan AI. Tanpa tata kelola 
yang baik, inisiatif BI dan AI berisiko menimbulkan bias, pelanggaran privasi, dan rendahnya 
kepercayaan pengguna \cite{khatri2010data, zwitter2014}.  

\subsection*{4. Organisasi dan Budaya}
Dimensi ini menilai sejauh mana organisasi telah menginternalisasi budaya berbasis data. 
Aspek yang diukur meliputi literasi data, kompetensi sumber daya manusia, peran \textit{data steward}, 
dan dukungan pimpinan puncak. Organisasi yang matang memiliki \textit{data-driven culture}, di mana 
pengambilan keputusan selalu berlandaskan bukti dan analitik, bukan intuisi semata \cite{davenport2010analytics, lim2023adoption}.

\subsection*{5. Strategi dan Nilai Bisnis}
Dimensi terakhir berfokus pada bagaimana BI dan AI mendukung pencapaian tujuan bisnis. 
Hal ini mencakup integrasi analitik ke dalam strategi perusahaan, pengukuran dampak bisnis 
(\textit{return on data and AI investment}), serta eksplorasi peluang monetisasi data. 
Semakin tinggi tingkat kematangan, semakin jelas keterkaitan antara inisiatif data dengan hasil bisnis 
yang terukur \cite{labreuche2020, ekambaram2021}.  

Dengan memahami kelima dimensi ini, organisasi dapat menilai posisi mereka secara komprehensif, 
serta mengidentifikasi prioritas peningkatan yang sesuai dengan kebutuhan dan konteks bisnis masing-masing.


\section{Tahapan Kematangan}

Model kematangan BI dan AI TDWI disusun dalam beberapa tahap evolusioner yang menggambarkan 
perjalanan organisasi dari kondisi awal yang minim pemanfaatan data hingga mencapai tahap 
terdepan dengan penerapan AI yang visioner. Setiap tahap ditandai dengan karakteristik, 
tantangan, serta peluang yang berbeda. Secara umum terdapat lima tahapan utama:  

\subsection*{1. Nascent}
Pada tahap ini, penggunaan BI dan AI masih sangat terbatas. Organisasi biasanya mengandalkan 
laporan manual, spreadsheet, dan data yang terfragmentasi dalam silo. Tidak ada strategi data 
yang jelas, dan tata kelola data hampir tidak diterapkan. Tantangan utama adalah kurangnya 
kesadaran manajemen puncak terhadap nilai strategis data \cite{russom2011, tdwi2013}.  

\subsection*{2. Early}
Organisasi mulai mengadopsi data warehouse atau dashboard sederhana untuk mendukung pelaporan. 
Namun, analitik yang dilakukan masih bersifat deskriptif dan berskala departemental. 
Pada tahap ini, organisasi menghadapi kesulitan dalam integrasi data antar unit bisnis, serta 
keterbatasan keterampilan analitik di kalangan pengguna \cite{schieder2011decision, raj2018}.  

\subsection*{3. Established}
Tahap ini ditandai dengan penerapan BI secara lebih luas di seluruh organisasi. 
Integrasi data mulai diperkuat melalui ETL dan tata kelola data yang lebih terstruktur. 
Analitik prediktif dan model machine learning sederhana mulai digunakan dalam proses bisnis 
seperti peramalan permintaan atau segmentasi pelanggan. Budaya berbasis data mulai tumbuh, 
meskipun adopsi AI masih terbatas pada pilot project \cite{yeoh2008biimplement, islam2017success}.  

\subsection*{4. Mature}
Pada tahap matang, organisasi telah mengadopsi analitik lanjutan secara operasional. 
Model AI dijalankan melalui kerangka MLOps yang memungkinkan deployment dan monitoring secara berkesinambungan. 
Tata kelola AI dan data governance diterapkan dengan baik, termasuk aspek etika, keamanan, 
dan kepatuhan. BI dan AI telah terintegrasi ke dalam pengambilan keputusan strategis, 
membawa dampak signifikan pada efisiensi, inovasi, dan daya saing \cite{liao2021business, lim2023adoption}.  

\subsection*{5. Advanced/Visionary}
Tahap tertinggi dicapai ketika organisasi tidak hanya mengadopsi AI, tetapi juga 
menjadi pelopor dalam inovasi berbasis data. Generative AI, augmented analytics, 
dan otomatisasi berbasis kecerdasan buatan digunakan untuk menciptakan produk, layanan, 
dan model bisnis baru. Data diperlakukan sebagai aset strategis yang dapat dimonetisasi, 
dan organisasi berperan sebagai pemimpin industri dalam pemanfaatan BI dan AI \cite{chen2021aibi, gantz2020futurebi}.  

Tahap-tahap kematangan ini bersifat evolusioner: organisasi perlu melewati fondasi awal 
sebelum melangkah ke tahap lanjutan. Dengan memahami posisi saat ini, organisasi dapat 
menyusun peta jalan (roadmap) untuk mempercepat transisi menuju tingkat kematangan berikutnya.

\section{Ikhtisar Tahap per Dimensi}

Setiap tahap kematangan BI dan AI dapat ditinjau dari lima dimensi utama yang telah dibahas: 
(1) data dan infrastruktur, (2) analitik dan teknologi, (3) tata kelola dan manajemen, 
(4) organisasi dan budaya, serta (5) strategi dan nilai bisnis. 
Ikhtisar berikut menunjukkan evolusi kelima dimensi tersebut pada setiap level kematangan.

\subsection{Nascent}
\begin{itemize}
	\item \textbf{Data dan Infrastruktur:} data tersebar dalam silo, tidak terintegrasi, dan kualitasnya sering diragukan. Penyimpanan berbasis spreadsheet mendominasi.
	\item \textbf{Analitik dan Teknologi:} analisis sebatas pelaporan manual; tidak ada adopsi BI formal, apalagi AI.
	\item \textbf{Tata Kelola dan Manajemen:} governance nyaris tidak ada, tidak ada standar metadata atau kebijakan keamanan data.
	\item \textbf{Organisasi dan Budaya:} literasi data rendah, keputusan didasarkan pada intuisi manajemen, bukan analitik.
	\item \textbf{Strategi dan Nilai Bisnis:} tidak ada strategi BI/AI; data belum dianggap sebagai aset strategis \cite{russom2011, tdwi2013}.
\end{itemize}

\subsection{Early}
\begin{itemize}
	\item \textbf{Data dan Infrastruktur:} mulai ada data warehouse atau sistem laporan sederhana, meskipun integrasi antar unit masih terbatas.
	\item \textbf{Analitik dan Teknologi:} dashboard digunakan untuk mendukung laporan deskriptif; eksplorasi analitik dilakukan secara ad-hoc.
	\item \textbf{Tata Kelola dan Manajemen:} awal penerapan kebijakan dasar data governance, tetapi belum konsisten.
	\item \textbf{Organisasi dan Budaya:} beberapa departemen mulai mengadopsi BI, namun budaya berbasis data belum melembaga.
	\item \textbf{Strategi dan Nilai Bisnis:} BI dianggap sebagai proyek TI, belum terintegrasi dalam strategi bisnis \cite{schieder2011decision, raj2018}.
\end{itemize}

\subsection{Established}
\begin{itemize}
	\item \textbf{Data dan Infrastruktur:} integrasi data lintas fungsi mulai berjalan melalui ETL dan data mart; kualitas data mulai diperhatikan.
	\item \textbf{Analitik dan Teknologi:} organisasi menggunakan analitik prediktif dasar (misalnya forecasting, segmentasi pelanggan).
	\item \textbf{Tata Kelola dan Manajemen:} tata kelola data lebih formal, termasuk pemilik data (data steward) dan mekanisme keamanan.
	\item \textbf{Organisasi dan Budaya:} literasi data meningkat, pelatihan mulai diberikan, dan dukungan manajemen senior terlihat.
	\item \textbf{Strategi dan Nilai Bisnis:} BI mulai diposisikan sebagai pendukung proses bisnis utama, bukan sekadar pelaporan \cite{yeoh2008biimplement, islam2017success}.
\end{itemize}

\subsection{Mature}
\begin{itemize}
	\item \textbf{Data dan Infrastruktur:} arsitektur data modern diterapkan, termasuk data lakehouse dan integrasi data real-time.
	\item \textbf{Analitik dan Teknologi:} model AI digunakan secara operasional; MLOps diterapkan untuk pengelolaan siklus hidup model.
	\item \textbf{Tata Kelola dan Manajemen:} governance mencakup AI, etika, serta kepatuhan regulasi seperti GDPR.
	\item \textbf{Organisasi dan Budaya:} budaya data-driven sudah melembaga; pimpinan aktif menggunakan BI/AI dalam pengambilan keputusan.
	\item \textbf{Strategi dan Nilai Bisnis:} analitik terhubung langsung dengan KPI bisnis, misalnya retensi pelanggan, optimasi supply chain, dan inovasi produk \cite{liao2021business, lim2023adoption}.
\end{itemize}

\subsection{Advanced/Visionary}
\begin{itemize}
	\item \textbf{Data dan Infrastruktur:} organisasi menjadi pelopor dalam adopsi teknologi analitik mutakhir, misalnya \textit{real-time digital twin} atau federated data platform.
	\item \textbf{Analitik dan Teknologi:} augmented analytics, generative AI, dan otomatisasi berbasis AI menjadi praktik standar; analitik digunakan tidak hanya untuk optimasi, tetapi juga inovasi.
	\item \textbf{Tata Kelola dan Manajemen:} tata kelola bersifat proaktif dan adaptif; organisasi memimpin dalam penerapan standar industri.
	\item \textbf{Organisasi dan Budaya:} karyawan di semua level memiliki literasi data tinggi; ada peran khusus seperti Chief AI Officer.
	\item \textbf{Strategi dan Nilai Bisnis:} data diperlakukan sebagai produk dan sumber pendapatan; organisasi memonetisasi data serta mengekspor keahlian BI/AI sebagai diferensiasi kompetitif \cite{chen2021aibi, gantz2020futurebi}.
\end{itemize}


\section{Dari Kondisi Kini ke Tahap Berikutnya}

Salah satu kekuatan utama model kematangan BI dan AI adalah kemampuannya membantu organisasi 
menentukan \textit{roadmap} pengembangan. Setiap organisasi berada pada titik yang berbeda dalam 
spektrum kematangan, dan tujuan utama bukanlah langsung mencapai tahap visioner, 
melainkan memastikan transisi yang realistis, terukur, dan berkelanjutan dari tahap saat ini 
menuju tahap berikutnya.

\subsection*{Identifikasi Posisi Saat Ini}
Langkah pertama adalah melakukan \textbf{penilaian diri (self-assessment)} untuk mengidentifikasi 
tahap kematangan saat ini berdasarkan lima dimensi inti: data, analitik, tata kelola, organisasi, 
dan strategi. Evaluasi ini dapat dilakukan melalui survei, wawancara, serta analisis infrastruktur 
dan proses bisnis yang ada \cite{alsai2023}.  

\subsection*{Menetapkan Prioritas Peningkatan}
Setelah posisi awal diketahui, organisasi perlu menetapkan \textbf{prioritas perbaikan} yang sesuai 
dengan kebutuhan bisnis. Misalnya:  
\begin{itemize}
	\item Dari \textit{Nascent} ke \textit{Early}: fokus pada konsolidasi data dalam data warehouse 
	dan membangun laporan standar.  
	\item Dari \textit{Early} ke \textit{Established}: memperluas analitik prediktif dan membangun 
	tata kelola data yang formal.  
	\item Dari \textit{Established} ke \textit{Mature}: mengintegrasikan MLOps dan memperkuat budaya 
	berbasis data.  
	\item Dari \textit{Mature} ke \textit{Advanced/Visionary}: mengembangkan inovasi berbasis AI 
	seperti augmented analytics dan monetisasi data.  
\end{itemize}

\subsection*{Faktor Pendorong Keberhasilan}
Transisi ke tahap berikutnya memerlukan kombinasi aspek teknis, manajerial, dan budaya. 
Beberapa faktor kunci yang sering disebut dalam literatur meliputi:
\begin{itemize}
	\item \textbf{Komitmen Manajemen Puncak:} sponsor eksekutif sangat menentukan keberhasilan 
	inisiatif BI dan AI \cite{yeoh2008biimplement}.  
	\item \textbf{Pengembangan Kapabilitas SDM:} literasi data, pelatihan AI, dan kehadiran peran 
	baru seperti \textit{data steward} dan \textit{ML engineer} sangat penting 
	\cite{lim2023adoption}.  
	\item \textbf{Investasi Teknologi:} migrasi ke platform modern seperti \textit{data lakehouse}, 
	penerapan cloud-native analytics, serta integrasi MLOps \cite{armbrust2021lakehouse}.  
	\item \textbf{Tata Kelola dan Kepatuhan:} keberhasilan transisi sangat dipengaruhi oleh 
	kemampuan organisasi menegakkan governance, baik dalam data maupun AI, sesuai standar 
	industri dan regulasi \cite{zwitter2014}.  
\end{itemize}

\subsection*{Mengukur Dampak dan Perbaikan Berkelanjutan}
Setiap langkah menuju tahap berikutnya harus diikuti dengan mekanisme \textbf{pengukuran dampak}. 
Indikator kinerja utama (KPI) mencakup: tingkat adopsi BI, penghematan biaya operasional, 
peningkatan akurasi prediksi, serta kontribusi analitik terhadap pendapatan. 
Dengan menerapkan prinsip \textit{continuous improvement}, organisasi dapat bergerak secara 
bertahap namun pasti menuju tingkat kematangan yang lebih tinggi \cite{ekambaram2021}.  

Secara keseluruhan, perjalanan dari kondisi kini ke tahap berikutnya bukanlah proses linier, 
melainkan sebuah siklus pembelajaran organisasi. Strategi yang efektif menuntut keseimbangan 
antara investasi teknologi, pengembangan sumber daya manusia, serta penerapan governance 
yang kuat untuk memastikan keberlanjutan dan dampak nyata bagi bisnis.

\section{Evaluasi Skor dan Interpretasi}

Evaluasi tingkat kematangan BI dan AI memerlukan pendekatan yang sistematis. 
Model TDWI memberikan kerangka untuk menilai posisi organisasi dengan 
menggunakan instrumen skoring pada tiap dimensi. Skor yang dihasilkan 
dapat digunakan untuk: (1) mengidentifikasi kesenjangan (gap analysis), 
(2) menentukan prioritas investasi, dan (3) menyusun peta jalan (roadmap) 
perubahan yang terukur.

\subsection{Skoring per Dimensi}

Setiap dimensi dalam model—data dan infrastruktur, analitik dan teknologi, 
tata kelola dan manajemen, organisasi dan budaya, serta strategi dan nilai bisnis—
diberikan skor berdasarkan indikator yang relevan. 
Pendekatan umum adalah menggunakan skala ordinal (misalnya 1–5) yang 
merepresentasikan lima tahap kematangan: \textit{Nascent, Early, Established, Mature}, 
dan \textit{Advanced/Visionary}.  

\begin{itemize}
	\item \textbf{Data dan Infrastruktur:} indikator meliputi tingkat integrasi data, 
	kualitas data, ketersediaan metadata, serta penggunaan platform modern 
	seperti data lakehouse.  
	Skor 1 menunjukkan dominasi spreadsheet dan silo data, sementara skor 5 
	menggambarkan data terintegrasi secara real-time dengan tata kelola kuat 
	\cite{kimball2013, armbrust2021lakehouse}.
	
	\item \textbf{Analitik dan Teknologi:} indikator mencakup cakupan analitik 
	(deskriptif hingga preskriptif), penggunaan AI/ML, serta operasionalisasi model 
	melalui MLOps.  
	Skor 1 identik dengan laporan manual, sedangkan skor 5 mencerminkan 
	penerapan augmented analytics, generative AI, dan otomatisasi cerdas 
	\cite{chen2012, chen2021aibi}.
	
	\item \textbf{Tata Kelola dan Manajemen:} penilaian dilakukan pada aspek kebijakan data, 
	kepatuhan regulasi, manajemen risiko, serta etika AI.  
	Skor 1 berarti tidak ada kebijakan formal, sedangkan skor 5 menunjukkan tata kelola 
	proaktif, adaptif, dan sesuai dengan standar global (misalnya GDPR, ISO 20547) 
	\cite{sadiq2017, gdpr2021bigdata}.
	
	\item \textbf{Organisasi dan Budaya:} indikator meliputi literasi data, 
	keterampilan SDM, keberadaan peran khusus seperti \textit{Chief Data Officer}, 
	dan tingkat dukungan manajemen puncak.  
	Skor 1 berarti keputusan berbasis intuisi, sementara skor 5 menunjukkan 
	budaya \textit{data-driven} yang melekat di seluruh level organisasi 
	\cite{davenport2010analytics, lim2023adoption}.
	
	\item \textbf{Strategi dan Nilai Bisnis:} indikator mencakup sejauh mana BI/AI 
	diintegrasikan dalam strategi bisnis, kontribusi terhadap KPI, serta eksplorasi 
	monetisasi data.  
	Skor 1 berarti BI hanya proyek TI, sedangkan skor 5 menunjukkan bahwa 
	data diperlakukan sebagai aset strategis yang menghasilkan nilai bisnis nyata 
	\cite{labreuche2020, ekambaram2021}.
\end{itemize}

Proses skoring ini dapat dilakukan melalui kuesioner terstruktur, wawancara 
dengan pemangku kepentingan, atau analisis dokumentasi kebijakan dan teknologi. 
Hasil skor per dimensi memberikan gambaran kuantitatif posisi organisasi 
dan menjadi dasar untuk evaluasi lebih lanjut.

\subsection{Membaca Hasil}

Setelah setiap dimensi dinilai dan diberikan skor, organisasi perlu melakukan interpretasi 
untuk memahami makna di balik angka tersebut. Proses membaca hasil bukan hanya melihat nilai 
numerik semata, tetapi menafsirkan implikasi strategisnya terhadap kondisi dan arah perkembangan BI dan AI.

\subsubsection*{Analisis Gap}
Langkah pertama adalah membandingkan skor aktual dengan skor target yang ingin dicapai. 
Perbedaan antara skor saat ini dan skor target disebut sebagai \textit{gap}.  
Sebagai contoh, jika dimensi \textit{Data dan Infrastruktur} saat ini berada pada skor 2 (Early) 
sementara target adalah skor 4 (Mature), maka diperlukan roadmap penguatan integrasi data, 
peningkatan kualitas, serta modernisasi platform untuk menjembatani kesenjangan tersebut \cite{tdwi2013, alsai2023}.

\subsubsection*{Profil Kematangan Organisasi}
Dengan menggabungkan skor dari seluruh dimensi, organisasi dapat membangun \textbf{profil kematangan}.  
Profil ini biasanya divisualisasikan menggunakan:
\begin{itemize}
	\item \textbf{Radar chart:} memperlihatkan distribusi skor setiap dimensi sehingga terlihat 
	area kuat dan area lemah.  
	\item \textbf{Heatmap:} menyoroti dimensi dengan gap paling besar untuk mempermudah prioritisasi.  
	\item \textbf{Benchmarking:} membandingkan skor organisasi dengan rata-rata industri atau kompetitor.  
\end{itemize}

\subsubsection*{Interpretasi Strategis}
Interpretasi skor harus dikaitkan dengan tujuan bisnis, bukan hanya kondisi teknis.  
Sebagai contoh:
\begin{itemize}
	\item Skor tinggi pada dimensi \textit{Analitik dan Teknologi} tetapi rendah pada 
	\textit{Tata Kelola} menunjukkan inovasi cepat namun dengan risiko kepatuhan.  
	\item Skor tinggi pada \textit{Strategi dan Nilai Bisnis} namun rendah pada 
	\textit{Organisasi dan Budaya} mengindikasikan adanya visi strategis 
	yang belum didukung oleh literasi data karyawan.  
\end{itemize}
Interpretasi ini membantu organisasi menghindari bias teknis dan tetap berorientasi 
pada penciptaan nilai bisnis \cite{ekambaram2021, labreuche2020}.

\subsubsection*{Prioritisasi Perbaikan}
Hasil evaluasi harus digunakan untuk menentukan area prioritas peningkatan.  
Misalnya, jika gap terbesar ada pada \textit{Organisasi dan Budaya}, 
maka investasi dalam literasi data, program pelatihan, dan pengembangan peran baru 
(\textit{data steward, ML engineer}) harus menjadi langkah utama.  

Dengan cara ini, skor kematangan tidak hanya berfungsi sebagai diagnosis, 
tetapi juga sebagai kompas yang mengarahkan transformasi BI dan AI secara berkelanjutan.

\section{Contoh Kasus Hasil Pengukuran di Bidang Bisnis}

Untuk memperjelas penerapan model kematangan BI dan AI, berikut adalah contoh kasus 
sebuah perusahaan ritel besar yang melakukan asesmen menggunakan model TDWI. 
Perusahaan ini memiliki jaringan toko fisik dan kanal e-commerce, serta mengelola 
jutaan transaksi pelanggan setiap bulannya.

\subsection*{Hasil Skor per Dimensi}
Berdasarkan kuesioner internal, wawancara dengan pimpinan unit, serta analisis sistem TI, 
diperoleh hasil skoring berikut:
\begin{itemize}
	\item \textbf{Data dan Infrastruktur:} Skor 3 (Established).  
	Data sudah terintegrasi dalam data warehouse, namun belum sepenuhnya 
	mendukung pemrosesan real-time.
	\item \textbf{Analitik dan Teknologi:} Skor 2 (Early).  
	Analisis masih berfokus pada laporan deskriptif dan dashboard, 
	belum ada integrasi machine learning dalam operasi sehari-hari.
	\item \textbf{Tata Kelola dan Manajemen:} Skor 2 (Early).  
	Kebijakan dasar mengenai kualitas data ada, namun belum ada mekanisme formal 
	untuk etika AI atau kepatuhan regulasi lintas negara.
	\item \textbf{Organisasi dan Budaya:} Skor 3 (Established).  
	Beberapa divisi mulai mengandalkan BI dalam pengambilan keputusan, 
	tetapi literasi data karyawan masih bervariasi.
	\item \textbf{Strategi dan Nilai Bisnis:} Skor 4 (Mature).  
	Pimpinan eksekutif menempatkan data sebagai prioritas strategis 
	dan memasukkannya dalam roadmap transformasi digital.
\end{itemize}

\subsection*{Interpretasi Hasil}
Profil kematangan di atas menunjukkan adanya ketidakseimbangan: 
strategi perusahaan sudah cukup matang (skor 4), tetapi kapabilitas analitik dan tata kelola 
masih tertinggal (skor 2). Hal ini mengindikasikan adanya “\textit{strategic aspiration gap}”—
visi bisnis tidak diimbangi dengan kesiapan teknis dan regulatif.  

Visualisasi radar chart dari skor ini memperlihatkan bentuk poligon yang tidak simetris, 
dengan titik lemah pada dimensi analitik dan governance.  
Hal ini memudahkan manajemen untuk menentukan area prioritas dalam jangka pendek.

\subsection*{Rekomendasi Strategis}

Berdasarkan hasil pengukuran, rekomendasi strategis berikut disusun dengan mempertimbangkan 
bidang (dimensi) dan skor saat ini:

\begin{itemize}
	\item \textbf{Analitik dan Teknologi (Skor 2 – Early):}  
	Fokus peningkatan pada pembentukan tim data science khusus, 
	integrasi machine learning untuk prediksi permintaan, serta 
	pengembangan sistem rekomendasi produk.  
	Hal ini penting untuk menaikkan level ke tahap \textit{Established}.
	
	\item \textbf{Tata Kelola dan Manajemen (Skor 2 – Early):}  
	Membentuk unit \textit{data governance}, menyusun kebijakan formal mengenai 
	kualitas data, serta mengadopsi standar internasional seperti GDPR untuk 
	perlindungan privasi.  
	Tujuannya agar organisasi siap menuju tahap \textit{Established} dengan 
	kepatuhan regulatif yang lebih kuat \cite{zwitter2014}.
	
	\item \textbf{Organisasi dan Budaya (Skor 3 – Established):}  
	Meluncurkan program literasi data lintas divisi, mengembangkan pelatihan 
	untuk karyawan non-teknis dalam menggunakan dashboard BI, serta memperkuat 
	peran Chief Data Officer (CDO).  
	Strategi ini akan mempercepat transisi ke tahap \textit{Mature} 
	\cite{davenport2010analytics}.
	
	\item \textbf{Data dan Infrastruktur (Skor 3 – Established):}  
	Melanjutkan modernisasi infrastruktur dengan migrasi ke platform 
	\textit{data lakehouse} agar mendukung pemrosesan real-time dan 
	analitik prediktif yang lebih cepat.  
	Langkah ini akan membantu organisasi bergerak menuju tahap \textit{Mature}.
	
	\item \textbf{Strategi dan Nilai Bisnis (Skor 4 – Mature):}  
	Memperluas inisiatif monetisasi data, seperti penawaran layanan analitik 
	bagi mitra bisnis atau personalisasi pengalaman pelanggan berbasis AI.  
	Rekomendasi ini ditujukan untuk membawa organisasi menuju tahap 
	\textit{Advanced/Visionary} \cite{liao2021business, ekambaram2021}.
\end{itemize}


\subsection*{Dampak Bisnis yang Diharapkan}
Dengan melaksanakan rekomendasi di atas, perusahaan diharapkan mampu:
\begin{enumerate}
	\item Meningkatkan akurasi prediksi inventori dan mengurangi biaya penyimpanan hingga 15\%.  
	\item Mengurangi risiko hukum terkait data privacy pada ekspansi pasar internasional.  
	\item Mempercepat pengambilan keputusan operasional di toko dan kanal digital.  
\end{enumerate}

Contoh ini menunjukkan bagaimana hasil pengukuran kematangan BI dan AI 
tidak berhenti pada diagnosis, tetapi dapat langsung diterjemahkan ke dalam 
intervensi strategis yang berdampak nyata bagi pertumbuhan bisnis \cite{liao2021business, ekambaram2021}.

\section{Ringkasan}

Model kematangan BI dan AI dari TDWI memberikan kerangka yang sistematis bagi organisasi 
untuk menilai posisi mereka dalam perjalanan transformasi digital. 
Melalui lima dimensi utama—data dan infrastruktur, analitik dan teknologi, tata kelola, 
organisasi dan budaya, serta strategi dan nilai bisnis—model ini membantu perusahaan 
memahami kekuatan yang sudah dimiliki sekaligus area yang masih perlu ditingkatkan. 
Pendekatan berbasis skoring memudahkan proses evaluasi, 
sementara interpretasi hasil memungkinkan organisasi melakukan analisis kesenjangan 
dan menyusun prioritas yang sejalan dengan kebutuhan bisnis. 

Secara keseluruhan, asesmen kematangan tidak hanya berfungsi sebagai diagnosis statis, 
tetapi juga sebagai \textit{roadmap} dinamis untuk mencapai tingkat kematangan yang lebih tinggi. 
Dengan membaca hasil skor per dimensi, organisasi dapat mengidentifikasi strategi peningkatan 
yang relevan—mulai dari penguatan infrastruktur data, adopsi teknologi analitik lanjutan, 
penerapan tata kelola yang ketat, hingga transformasi budaya menjadi lebih \textit{data-driven}. 
Pendekatan ini memungkinkan bisnis untuk secara bertahap bergerak menuju tahap 
\textit{Advanced/Visionary}, di mana BI dan AI bukan sekadar fungsi pendukung, 
melainkan penggerak utama inovasi dan penciptaan nilai kompetitif yang berkelanjutan.

