\chapter{Pembersihan dan Persiapan Data}

\section{Pendahuluan}

Pembersihan dan persiapan data merupakan langkah dasar dalam setiap proses analisis data. Data yang berkualitas tinggi dan dipersiapkan dengan baik memastikan bahwa analisis, pemodelan, dan pengambilan keputusan selanjutnya akan akurat dan dapat diandalkan. Tanpa pembersihan dan persiapan yang tepat, bahkan analisis dan visualisasi tingkat lanjut pun dapat menghasilkan wawasan yang menyesatkan \cite{kim2020data}.

Pembersihan data melibatkan identifikasi dan perbaikan kesalahan atau inkonsistensi dalam dataset, seperti nilai yang hilang, outlier, atau data duplikat \cite{rahm2000data}. Sementara itu, persiapan data mencakup aktivitas yang lebih luas termasuk transformasi, integrasi, dan pengorganisasian data ke dalam bentuk yang dapat digunakan untuk analisis \cite{kotu2014predictive}.

Dalam lingkungan bisnis, proses-proses ini sangat penting karena data sering berasal dari berbagai sumber dengan format, struktur, dan tingkat kualitas yang berbeda-beda. Bagi manajemen dan pengambil keputusan, memahami pentingnya pembersihan dan persiapan data membantu mereka menghargai waktu dan upaya yang diperlukan sebelum data dapat memberikan wawasan yang bermakna \cite{gandomi2015beyond}. Bab ini memperkenalkan konsep utama, teknik, dan tantangan dalam pembersihan dan persiapan data untuk analisis yang efektif dan pengambilan keputusan bisnis strategis.

\section{Memahami Kualitas Data}

\subsection{Definisi dan Pentingnya}

\begin{figure}[h]
	\centering
	\begin{tikzpicture}
		\begin{polaraxis}[
			width=7cm,
			height=7cm,
			grid=none, % turn off default grid to customise order
			xtick=data,
			xticklabels={\textbf{Akurasi}, \textbf{Kelengkapan}, \textbf{Konsistensi}, \textbf{Ketepatan Waktu}, \textbf{Validitas}, \textbf{Keunikan}},
			tick label style={font=\small},
			ylabel=,
			ytick=\empty,
			ymin=0, ymax=1,
			]
			% Draw filled polygon first
			\addplot[fill=blue!30, draw=blue] coordinates {
				(0,1)
				(60,1)
				(120,1)
				(180,1)
				(240,1)
				(300,1)
				(360,1)
			};
			
			% Add lines from centre to each corner
			\draw[black] (axis cs:0,0) -- (axis cs:0,1);
			\draw[black] (axis cs:60,0) -- (axis cs:60,1);
			\draw[black] (axis cs:120,0) -- (axis cs:120,1);
			\draw[black] (axis cs:180,0) -- (axis cs:180,1);
			\draw[black] (axis cs:240,0) -- (axis cs:240,1);
			\draw[black] (axis cs:300,0) -- (axis cs:300,1);
			
		\end{polaraxis}
	\end{tikzpicture}
	\caption{Dimensi Kualitas Data: Akurasi, Kelengkapan, Konsistensi, Ketepatan Waktu, Validitas, Keunikan}
	\label{fig:dimensi-kualitas-data}
\end{figure}





Kualitas data mengacu pada sejauh mana data akurat, lengkap, dapat diandalkan, dan relevan untuk tujuan penggunaannya \cite{wang1996beyond}. Dalam konteks bisnis, data yang berkualitas tinggi memungkinkan pengambilan keputusan yang efektif, efisiensi operasional, dan perencanaan strategis. Sebaliknya, data dengan kualitas rendah dapat menyebabkan analisis yang salah, keputusan bisnis yang keliru, dan kerugian finansial \cite{redman1998impact}.

Dimensi utama kualitas data meliputi akurasi, kelengkapan, konsistensi, ketepatan waktu, dan validitas \cite{pipino2002data}. Akurasi menunjukkan apakah data menggambarkan objek dunia nyata dengan benar. Kelengkapan mengacu pada sejauh mana semua data yang dibutuhkan tersedia. Konsistensi memastikan data sama di berbagai dataset atau sistem. Ketepatan waktu menilai apakah data terbaru, sedangkan validitas terkait dengan kesesuaian data terhadap format atau aturan yang ditetapkan.

Memahami kualitas data sangat penting bagi mahasiswa manajemen karena hal ini menunjukkan risiko yang muncul jika bergantung pada data tanpa menilai kualitasnya terlebih dahulu \cite{strong1997data}. Sebagai contoh, kampanye pemasaran yang menggunakan informasi pelanggan yang tidak akurat mungkin gagal menjangkau audiens yang dituju, sehingga mengakibatkan pemborosan sumber daya dan hilangnya peluang. Oleh karena itu, kesadaran akan prinsip kualitas data membantu pemimpin bisnis mengajukan pertanyaan yang tepat ketika disajikan laporan analitik atau rekomendasi strategis yang dihasilkan dari data.

\subsection{Dimensi Kualitas Data}

Kualitas data merupakan konsep multidimensi yang mencakup beberapa atribut yang menentukan apakah data layak digunakan. Memahami dimensi-dimensi ini membantu organisasi menilai dan meningkatkan data mereka secara efektif \cite{lee2006data}.

Dimensi utama kualitas data adalah:

\textbf{1. Akurasi.} Menggambarkan seberapa tepat data merepresentasikan nilai atau kejadian di dunia nyata. Misalnya, tanggal lahir pelanggan yang tercatat harus sesuai dengan tanggal lahir sebenarnya. Data yang tidak akurat dapat menyebabkan analisis yang salah dan keputusan yang buruk \cite{batini2009methodologies}.

\textbf{2. Kelengkapan.} Menunjukkan apakah semua data yang dibutuhkan tersedia. Data yang hilang dapat membatasi ruang lingkup analisis. Misalnya, jika data pendapatan hilang untuk banyak pelanggan, model segmentasi atau penilaian kredit menjadi tidak dapat diandalkan.

\textbf{3. Konsistensi.} Mengacu pada kesamaan data di berbagai dataset atau sistem. Contohnya, jika alamat pelanggan tercatat berbeda di sistem penjualan dan penagihan, maka dapat terjadi kesalahan operasional \cite{madnick2009overview}.

\textbf{4. Ketepatan Waktu.} Menjamin bahwa data selalu diperbarui. Data yang sudah usang mengurangi relevansi. Sebagai contoh, menggunakan data pasar tahun lalu untuk menentukan harga saat ini dapat menyebabkan ketidakmampuan bersaing.

\textbf{5. Validitas.} Mengindikasikan apakah data sesuai dengan format, aturan, atau standar yang ditentukan. Misalnya, kolom kode pos harus berisi kode pos yang valid dalam wilayah yang relevan.

\textbf{6. Keunikan.} Dimensi ini memastikan bahwa setiap entitas dicatat hanya satu kali tanpa duplikasi. Rekaman duplikat dapat mengacaukan hasil analisis, seperti menggandakan jumlah pelanggan dalam laporan.

Bagi manajer bisnis, dimensi-dimensi ini menyoroti area spesifik yang perlu diperiksa saat menilai kualitas data, sehingga mendukung tata kelola dan pengambilan keputusan yang lebih baik \cite{caballero2014quality}.

\section{Gambaran Umum Pembersihan dan Persiapan Data}

\subsection{Definisi Pembersihan Data}

\begin{figure}[h]
	\centering
	\begin{tikzpicture}[
		node distance=.5cm and .5cm,
		box/.style={rectangle, draw=black, fill=blue!10, rounded corners, minimum width=5cm, minimum height=1cm, align=center},
		subbox/.style={rectangle, draw=black, fill=green!20, rounded corners, minimum width=3cm, minimum height=0.8cm, align=center},
		arrow/.style={->, thick}
		]
		
		% Main Data Preparation box at centre
		\node[box] (prep) {\textbf{Persiapan Data}\\(Data Preparation)};
		
		% Two components above shifted left and right
		\node[subbox, above left=1cm and -1cm of prep] (clean) {\textbf{Pembersihan Data}\\(Cleaning)};
		\node[subbox, above right=1cm and -1cm of prep] (integration) {\textbf{Integrasi Data}\\(Integration)};
		
		% Three components below
		\node[subbox, below=1cm of prep] (transform) {\textbf{Transformasi Data}\\(Transformation)};
		\node[subbox, left=of transform] (reduction) {\textbf{Reduksi Data}\\(Reduction)};
		\node[subbox, right=of transform] (formatting) {\textbf{Pemformatan Data}\\(Formatting)};
		
		% Inclusion arrows from Preparation to subprocesses
		\draw[arrow] (prep) -- (clean);
		\draw[arrow] (prep) -- (integration);
		\draw[arrow] (prep) -- (transform);
		\draw[arrow] (prep) -- (reduction);
		\draw[arrow] (prep) -- (formatting);
		
	\end{tikzpicture}
	\caption{Struktur Persiapan Data dan Komponennya}
	\label{fig:struktur-persiapan-data}
\end{figure}




Pembersihan data adalah proses mengidentifikasi dan memperbaiki kesalahan, inkonsistensi, atau ketidakakuratan dalam dataset untuk meningkatkan kualitas dan keandalannya dalam analisis serta pengambilan keputusan \cite{kurgan2006survey}. Ini mencakup penanganan nilai yang hilang, penghapusan data duplikat, perbaikan kesalahan ketik, dan penyelesaian inkonsistensi data.

Dalam konteks bisnis, pembersihan data memastikan bahwa informasi yang digunakan untuk perencanaan strategis, pemantauan kinerja, dan analisis pelanggan dapat dipercaya. Misalnya, jika alamat email pelanggan mengandung kesalahan ketik atau sudah tidak aktif, kampanye pemasaran mungkin gagal menjangkau audiens yang dituju, menyebabkan pemborosan sumber daya dan hilangnya peluang \cite{rahm2000dataquality}.

Pembersihan data sering dianggap sebagai langkah yang paling memakan waktu dalam proyek analisis data, dengan studi menunjukkan bahwa proses ini dapat menyita hingga 80\% waktu analis atau ilmuwan data \cite{dasu2003exploratory}. Namun, investasi ini sangat penting karena keputusan yang didasarkan pada data yang kotor dapat menimbulkan kesalahan mahal, seperti prakiraan penjualan yang salah, alokasi anggaran yang keliru, atau risiko kepatuhan akibat pelaporan yang tidak akurat.

Pada akhirnya, pembersihan data menjadi fondasi bagi semua proses persiapan, integrasi, dan analisis data berikutnya, memungkinkan organisasi mendapatkan wawasan yang bermakna dan dapat diandalkan dari aset data mereka.

\subsection{Definisi Persiapan Data}

Persiapan data adalah proses mengubah data mentah menjadi bentuk yang sesuai untuk analisis dan pengambilan keputusan. Proses ini meliputi serangkaian langkah seperti pembersihan data, integrasi, transformasi, reduksi, dan pemformatan agar data siap digunakan dalam model analitik atau pelaporan \cite{pyle1999data}.

Dalam konteks bisnis, persiapan data memungkinkan analis mengubah data yang tersebar, inkonsisten, atau tidak terstruktur menjadi dataset terstruktur yang dapat menjawab pertanyaan bisnis secara efektif. Misalnya, menyiapkan data penjualan dapat melibatkan penggabungan catatan dari beberapa toko, standarisasi kode produk, agregasi penjualan mingguan, dan pemformatan ulang tanggal ke format yang konsisten.

Persiapan data yang efektif meningkatkan efisiensi dan akurasi tugas analitik, mendukung pengambilan keputusan yang tepat, pemantauan kinerja, dan perencanaan strategis \cite{zeller2014data}.

\subsection{Hubungan antara Pembersihan dan Persiapan}

Pembersihan data dan persiapan data adalah konsep yang saling berkaitan tetapi tidak sama. Pembersihan data adalah \textbf{subproses dalam persiapan data} yang berfokus pada identifikasi dan perbaikan kesalahan, inkonsistensi, serta ketidakakuratan data \cite{kelleher2015fundamentals}. Ini memastikan bahwa data memiliki kualitas yang baik.

Sementara itu, persiapan data adalah proses yang lebih luas. Ini mencakup pembersihan data serta aktivitas lain seperti integrasi data (menggabungkan dataset dari berbagai sumber), transformasi data (mengubah format atau struktur data agar sesuai dengan kebutuhan analisis), reduksi data (memilih variabel yang relevan atau mengagregasi data), dan pemformatan data (memastikan data sesuai dengan format input yang dibutuhkan model atau alat) \cite{han2011data}.

Secara sederhana, \textbf{pembersihan data memastikan data benar dan andal, sedangkan persiapan data memastikan data siap digunakan untuk tujuan tertentu}. Keduanya penting untuk analisis yang akurat dan pengambilan keputusan bisnis yang berhasil.

\subsection{Teknik Utama Pembersihan Data}


\begin{table}[h]
	\centering
	\renewcommand{\arraystretch}{1.2}
	\begin{tabular}{|p{0.3\textwidth}|p{0.6\textwidth}|}
		\hline
		\textbf{Teknik Pembersihan Data} & \textbf{Contoh Implementasi} \\
		\hline
		Menangani Nilai Hilang & Imputasi umur pelanggan yang hilang dengan rata-rata kelompok umur serupa \\
		\hline
		Menangani Outlier & Log transformasi pada data pendapatan ekstrem untuk mengurangi pengaruhnya \\
		\hline
		Normalisasi Data & Min-Max scaling pada kolom penjualan untuk skala 0-1 sebelum analisis \\
		\hline
		Menghapus Duplikasi & Menghapus entri duplikat pelanggan saat menggabungkan data dari dua toko \\
		\hline
		Memperbaiki Kesalahan Input Data & Koreksi kesalahan ketik pada kode produk atau standarisasi format tanggal \\
		\hline
	\end{tabular}
	\caption{Teknik Utama Pembersihan Data dan Contoh Implementasi}
	\label{tab:teknik-pembersihan-data}
\end{table}



Pembersihan data melibatkan berbagai teknik untuk memastikan data akurat, konsisten, dan siap dianalisis. Pemilihan teknik tergantung pada sifat data dan jenis masalah yang ditemukan \cite{rahm2000data}.

\textbf{1. Menangani Nilai yang Hilang.}  
Nilai yang hilang umum ditemukan dalam dataset bisnis akibat kesalahan manusia, keterbatasan sistem, atau proses yang tidak lengkap. Teknik untuk menanganinya meliputi:

- \emph{Penghapusan}, yaitu menghapus baris dengan nilai hilang jika jumlahnya sedikit dan tidak menimbulkan bias pada dataset.
- \emph{Imputasi}, yaitu mengganti nilai hilang dengan estimasi seperti rata-rata, median, modus, atau nilai yang diprediksi variabel lain.

Sebagai contoh, jika data umur pelanggan hilang dalam dataset, nilai tersebut dapat diganti dengan rata-rata umur pelanggan serupa untuk menghindari kehilangan data.

\textbf{2. Menangani Outlier.}  
Outlier adalah data yang sangat berbeda dari pengamatan lain. Ini bisa berasal dari kesalahan input data atau kasus ekstrim yang valid. Teknik yang umum digunakan meliputi:

- \emph{Investigasi}, untuk menentukan apakah outlier adalah kesalahan atau nilai ekstrim yang sah.
- \emph{Penghapusan atau transformasi}, jika outlier adalah kesalahan maka diperbaiki atau dihapus, sedangkan outlier yang valid dapat ditransformasi menggunakan penskalaan atau log transformasi untuk mengurangi dampaknya dalam analisis \cite{barnett1994outliers}.

\textbf{3. Normalisasi Data.}  
Normalisasi menskalakan data numerik ke rentang umum, seperti 0 hingga 1, agar variabel dengan skala besar tidak mendominasi model analitik. Metode umum termasuk:

- \emph{Min-Max scaling}, untuk mengubah skala data ke rentang tertentu.
- \emph{Standarisasi Z-score}, untuk menempatkan data di sekitar rata-rata dengan standar deviasi satu \cite{jain2005score}.

\textbf{4. Menghapus Duplikasi.}  
Rekaman duplikat sering muncul saat menggabungkan dataset dari berbagai sumber. Menghapusnya memastikan setiap entitas hanya tercatat satu kali, mencegah perhitungan ganda dalam analisis seperti jumlah pelanggan atau volume penjualan.

\textbf{5. Memperbaiki Kesalahan Input Data.}  
Kesalahan ketik atau ketidakkonsistenan format (misalnya format tanggal, kode produk) diperbaiki dengan standarisasi format atau validasi menggunakan sumber referensi resmi.

Penerapan teknik pembersihan data ini meningkatkan kualitas dan keandalan data, sehingga keputusan bisnis didasarkan pada informasi yang akurat dan dapat dipercaya \cite{oliveira2005data}.

\section{Integrasi Data}


\begin{figure}[h]
	\centering
	\begin{tikzpicture}[
		node distance=0.5cm and 0.8cm,
		box/.style={rectangle, draw=black, fill=orange!20, rounded corners, minimum width=4.5cm, minimum height=1cm, align=left, font=\small},
		sourcebox/.style={rectangle, draw=black, fill=blue!20, rounded corners, minimum width=4.5cm, minimum height=0.5cm, align=left, font=\small},
		arrow/.style={->, thick}
		]
		
		% Source systems node (different color)
		\node[sourcebox] (sources) {\textbf{Sumber Data}\\Berbagai sistem dan format};
		
		% ETL
		\node[box, above right=.1cm and .7cm of sources] (etl) {\textbf{ETL}\\Extract, Transform, Load};
		\draw[arrow] (sources) -- (etl);
		\node[box, right=of etl] (dw1) {\textbf{Data Warehouse}\\Repositori terpusat};
		\draw[arrow] (etl) -- (dw1);
		
		% ELT
		\node[box, below right=.1cm and .7cm of sources] (elt) {\textbf{ELT}\\Extract, Load, Transform};
		\draw[arrow] (sources) -- (elt);
		\node[box, right=of elt] (dw2) {\textbf{Data Lake/Warehouse}\\Transformasi di target};
		\draw[arrow] (elt) -- (dw2);
		
		% Data Federation
		\node[box, above=2cm of sources] (federation) {\textbf{Federasi Data}\\Integrasi virtual real-time};
		\draw[arrow] (sources) -- (federation);
		
		% Data Warehousing (direct)
		\node[box, right=1.3cm of federation] (warehouse) {\textbf{Data Warehousing}\\Konsolidasi data terstruktur};
		\draw[arrow] (sources) -- (warehouse.west);
		
		% Application-Based Integration
		\node[box, below=2cm of sources] (appint) {\textbf{Integrasi Aplikasi}\\API atau middleware};
		\draw[arrow] (sources) -- (appint);
		
		% Manual Integration
		\node[box, right=1.3cm of appint] (manual) {\textbf{Integrasi Manual}\\Spreadsheet atau copy-paste};
		\draw[arrow] (sources) -- (manual.west);
		
	\end{tikzpicture}
	\caption{Teknik Integrasi Data: ETL, ELT, Federasi, Warehousing, Integrasi Aplikasi, dan Manual}
	\label{fig:teknik-integrasi-data}
\end{figure}




\subsection{Definisi dan Tujuan}


Integrasi data adalah proses menggabungkan data dari berbagai sumber untuk memberikan pandangan yang terpadu dan konsisten dalam analisis serta pengambilan keputusan \cite{lenzerini2002data}. Ini melibatkan penggabungan dataset yang mungkin memiliki format, struktur, atau definisi berbeda ke dalam satu dataset yang dapat digunakan secara efisien.

Dalam konteks bisnis, integrasi data penting karena organisasi sering menyimpan informasi di banyak sistem. Misalnya, data pelanggan dapat tersebar di database penjualan, catatan layanan pelanggan, dan platform pemasaran. Tanpa integrasi, menganalisis perjalanan pelanggan secara menyeluruh atau membuat segmentasi yang akurat menjadi sulit \cite{doan2012principles}.

Tujuan utama integrasi data adalah:

- \textbf{Pandangan Holistik.} Integrasi memungkinkan organisasi melihat dan menganalisis semua data relevan dalam satu tempat untuk mendapatkan wawasan menyeluruh.
- \textbf{Pengambilan Keputusan yang Lebih Baik.} Dengan menggabungkan data dari berbagai sumber, manajer dapat membuat keputusan yang lebih tepat berdasarkan informasi yang lengkap dan konsisten.
- \textbf{Efisiensi Operasional.} Data terintegrasi mengurangi duplikasi kerja, menyederhanakan proses, dan meningkatkan komunikasi antar departemen.
- \textbf{Konsistensi Data.} Integrasi memastikan data konsisten di seluruh sistem, menghindari laporan yang bertentangan atau keputusan yang dibuat berdasarkan informasi yang tidak lengkap.

Sebagai contoh, mengintegrasikan data inventaris dari sistem gudang dengan data penjualan memungkinkan prakiraan permintaan yang akurat dan manajemen stok yang efisien. Demikian pula, menggabungkan umpan balik pelanggan dengan riwayat pembelian membantu mengidentifikasi perbaikan layanan dan peluang pengembangan produk \cite{hernandez1995merge}.

Secara keseluruhan, integrasi data adalah proses penting dalam mempersiapkan data untuk analisis, memungkinkan organisasi memanfaatkan aset data mereka secara efektif untuk tujuan strategis dan operasional.

\subsection{Teknik Integrasi Data}

Berbagai teknik digunakan untuk mengintegrasikan data dari banyak sumber secara efektif. Pemilihan teknik tergantung pada sifat data, sistem yang terlibat, dan tujuan bisnis \cite{singh2005survey}.

\textbf{1. ETL (Extract, Transform, Load).}  
Ini adalah pendekatan integrasi paling umum di mana data:

- \emph{Diekstraksi} dari berbagai sistem sumber.
- \emph{Ditranformasi} ke dalam format atau struktur yang seragam.
- \emph{Dimuat} ke dalam sistem target seperti data warehouse.

Misalnya, data penjualan dari cabang regional diekstraksi, ditransformasi ke mata uang dan format standar, lalu dimuat ke database pelaporan pusat perusahaan \cite{vassiliadis2002etl}.

\textbf{2. ELT (Extract, Load, Transform).}  
Dalam pendekatan ini, data pertama-tama diekstraksi dan dimuat ke sistem target (misalnya data lake) lalu ditransformasi di dalam sistem tersebut. Pendekatan ini cocok bila sistem target memiliki kapasitas pemrosesan tinggi dan sering digunakan dalam lingkungan big data.

\textbf{3. Federasi Data (Integrasi Virtual).}  
Alih-alih memindahkan data secara fisik, teknik ini menggunakan lapisan virtual untuk mengintegrasikan data dari berbagai sumber secara real-time. Query dijalankan melintasi banyak database dan hasil ditampilkan seolah-olah data telah terintegrasi. Teknik ini cocok jika memerlukan data terbaru tanpa menduplikasi penyimpanan \cite{halevy2001answering}.

\textbf{4. Data Warehousing.}  
Data warehouse mengkonsolidasikan data dari berbagai sumber ke dalam repositori terpusat yang dirancang untuk analisis dan pelaporan. Data dibersihkan, ditransformasi, dan disimpan dalam format terstruktur, memungkinkan query yang efisien untuk business intelligence.

\textbf{5. Integrasi Berbasis Aplikasi.}  
Dalam pendekatan ini, integrasi dilakukan melalui aplikasi yang berkomunikasi langsung menggunakan API atau perangkat lunak integrasi. Misalnya, integrasi sistem CRM dan ERP sehingga pesanan pelanggan di CRM secara otomatis memperbarui level inventaris di ERP.

\textbf{6. Integrasi Manual.}  
Terkadang, integrasi dilakukan secara manual, terutama dalam proyek skala kecil atau dataset ad-hoc, menggunakan alat seperti spreadsheet untuk menggabungkan dan menyelaraskan data. Namun, cara ini rawan kesalahan manusia dan tidak dapat diskalakan untuk kebutuhan perusahaan.

Setiap teknik memiliki kelebihan dan keterbatasannya. ETL efektif untuk integrasi data terstruktur, sedangkan federasi data ideal untuk kebutuhan real-time tanpa duplikasi data. Memahami teknik-teknik ini membantu manajer berkolaborasi secara efektif dengan tim IT dalam merencanakan proyek data yang mendukung tujuan strategis organisasi \cite{hasselbring2000information}.


\subsection{Tantangan dalam Integrasi Data}

Meskipun integrasi data memberikan manfaat signifikan, proses ini juga menghadirkan berbagai tantangan yang harus diatasi organisasi agar implementasinya berhasil \cite{alexe2006cleaning}.

\textbf{1. Heterogenitas Data.}  
Data sering berasal dari berbagai sumber dengan format, struktur, dan standar yang berbeda. Misalnya, data pelanggan di satu sistem mencatat tanggal sebagai “DD/MM/YYYY” sedangkan sistem lain menggunakan “MM-DD-YYYY”. Menyatukan perbedaan ini memerlukan transformasi dan standarisasi yang cermat untuk menghindari kesalahan dalam analisis.

\textbf{2. Perbedaan Skema dan Semantik.}  
Sistem yang berbeda mungkin menggunakan konvensi penamaan dan makna yang berbeda untuk data serupa. Misalnya, satu database melabeli kolom sebagai “CustomerID” sedangkan database lain menggunakan “CustNum”. Selain itu, istilah seperti “sales” bisa berarti penjualan kotor di satu dataset dan penjualan bersih di dataset lain, yang dapat menimbulkan kebingungan dan kesalahan integrasi \cite{doan2003reconciling}.

\textbf{3. Masalah Kualitas Data.}  
Integrasi dapat memperbesar masalah kualitas data yang sudah ada, seperti nilai yang hilang, duplikasi, atau ketidakakuratan. Menggabungkan data tanpa pembersihan dapat menghasilkan output yang tidak dapat diandalkan dan keputusan bisnis yang salah.

\textbf{4. Skalabilitas dan Kinerja.}  
Mengintegrasikan volume data yang besar, terutama secara real-time, dapat membebani sumber daya sistem. Memastikan bahwa proses integrasi dapat diskalakan tanpa memperlambat operasi adalah tantangan teknis utama \cite{nash2019big}.

\textbf{5. Masalah Keamanan dan Privasi.}  
Integrasi data sering melibatkan pemindahan dan akses informasi sensitif antar sistem. Memastikan kepatuhan terhadap regulasi privasi data (seperti GDPR) dan menjaga keamanan data merupakan pertimbangan penting dalam proyek integrasi \cite{zhang2019security}.

\textbf{6. Tantangan Organisasi dan Tata Kelola.}  
Selain masalah teknis, integrasi data memerlukan kerja sama antar departemen, kesepakatan tentang definisi dan kepemilikan data, serta penetapan kebijakan tata kelola data untuk menjaga konsistensi dan akuntabilitas.

Mengatasi tantangan ini memerlukan kombinasi solusi teknis, kebijakan organisasi yang jelas, dan kolaborasi antara tim bisnis dan TI untuk mencapai integrasi data yang andal, efisien, dan aman demi mendukung tujuan strategis.

\section{Menjaga Konsistensi Data}

\subsection{Definisi dan Pentingnya}

Konsistensi data mengacu pada keseragaman dan koherensi data di berbagai dataset, sistem, atau proses dalam suatu organisasi. Ini memastikan bahwa nilai data tidak bertentangan antar sistem dan tetap akurat serta dapat diandalkan seiring waktu \cite{chen2012data}.

Sebagai contoh, jika alamat pelanggan diperbarui di database penjualan tetapi tidak di sistem pengiriman, maka akan terjadi inkonsistensi yang dapat menyebabkan kesalahan pengiriman dan menurunkan kepuasan pelanggan. Demikian pula, dalam pelaporan keuangan, jika angka pendapatan berbeda antara database akuntansi dan penjualan karena pembaruan yang tidak konsisten, hal ini dapat menghasilkan keputusan strategis yang salah dan masalah kepatuhan.

Pentingnya konsistensi data dalam bisnis meliputi:

- \textbf{Pengambilan Keputusan yang Lebih Baik.} Data yang konsisten memastikan keputusan manajemen didasarkan pada informasi yang seragam dan dapat diandalkan, sehingga mengurangi risiko akibat data yang bertentangan.

- \textbf{Efisiensi Operasional.} Proses seperti pemenuhan pesanan, penagihan, dan layanan pelanggan bergantung pada data yang konsisten untuk berjalan lancar tanpa penundaan atau kesalahan.

- \textbf{Kepatuhan Regulasi.} Banyak industri yang memerlukan pelaporan data yang konsisten di seluruh sistem untuk memenuhi standar hukum dan regulasi. Inkonsistensi dapat menyebabkan pelanggaran kepatuhan dan denda finansial \cite{rahm2000dataquality}.

- \textbf{Pengalaman Pelanggan yang Lebih Baik.} Data pelanggan yang konsisten di seluruh titik kontak memastikan komunikasi dan layanan yang dipersonalisasi serta akurat, meningkatkan kepuasan dan loyalitas pelanggan.

Menjaga konsistensi data memerlukan kebijakan tata kelola data yang efektif, proses integrasi yang baik, dan validasi berkala untuk memastikan semua sistem mencerminkan data yang sama dan benar \cite{ottoo2013data}.

\subsection{Metode untuk Menjaga Konsistensi}

\begin{table}[h]
	\centering
	\caption{Metode untuk Menjaga Konsistensi Data}
	\begin{tabular}{|p{0.2\textwidth}|p{0.70\textwidth}|}
		\hline
		\textbf{Metode} & \textbf{Deskripsi} \\
		\hline
		Master Data Management (MDM) & Membuat sumber tunggal yang otoritatif untuk data penting (pelanggan, produk, pemasok) agar semua departemen dan sistem merujuk pada data master yang sama, mengurangi duplikasi dan konflik data \cite{otto2011mdm}. \\
		\hline
		Aturan Validasi Data & Menerapkan aturan validasi saat entri dan integrasi data untuk memastikan data sesuai standar sebelum masuk sistem, misalnya format tanggal yang konsisten atau kode produk yang valid. \\
		\hline
		Batasan Integritas Referensial & Menjaga hubungan antar tabel dalam database relasional tetap konsisten, misalnya ID pelanggan pada tabel pesanan harus cocok dengan tabel pelanggan, mencegah rekaman yatim atau referensi tidak valid. \\
		\hline
		Sinkronisasi Data & Memastikan pembaruan data di satu sistem tercermin di sistem lain melalui pembaruan batch, sinkronisasi real-time menggunakan API, atau middleware integration tools. \\
		\hline
		Audit dan Rekonsiliasi Berkala & Melakukan audit data secara berkala untuk mendeteksi inkonsistensi antar sistem dan merekonsiliasi data, seperti mencocokkan transaksi keuangan antara sistem akuntansi dan penjualan \cite{laurila1999data}. \\
		\hline
		Kebijakan Tata Kelola Data & Menetapkan kebijakan kepemilikan data, tanggung jawab pembaruan, dan prosedur manajemen perubahan untuk memastikan akuntabilitas dan konsistensi pengelolaan data di seluruh organisasi. \\
		\hline
	\end{tabular}
	\label{tab:metode-konsistensi-data}
\end{table}

Menjaga konsistensi data memerlukan penerapan metode sistematis dan praktik organisasi untuk memastikan data tetap seragam di seluruh sistem dan proses. Metode utama meliputi \cite{watson2009data}:

\textbf{1. Master Data Management (MDM).}  
MDM melibatkan pembuatan sumber tunggal yang otoritatif untuk data bisnis penting seperti pelanggan, produk, atau pemasok. Ini memastikan semua departemen dan sistem merujuk pada rekaman master yang sama, mengurangi duplikasi dan konflik data \cite{otto2011mdm}.

\textbf{2. Aturan Validasi Data.}  
Menerapkan aturan validasi saat entri data dan integrasi memastikan data memenuhi standar yang diperlukan sebelum masuk ke sistem. Misalnya, penerapan format tanggal yang konsisten atau kode produk yang valid di seluruh dataset mencegah inkonsistensi.

\textbf{3. Batasan Integritas Referensial.}  
Dalam database relasional, integritas referensial memastikan hubungan antar tabel tetap konsisten. Misalnya, setiap rekaman pesanan yang merujuk ID pelanggan harus cocok dengan rekaman pelanggan yang ada. Ini mencegah adanya rekaman yatim atau referensi yang tidak valid.

\textbf{4. Sinkronisasi Data.}  
Sinkronisasi data antar sistem memastikan pembaruan di satu sistem tercermin di sistem lain. Tekniknya termasuk pembaruan batch, sinkronisasi real-time melalui API, atau alat integrasi middleware.

\textbf{5. Audit dan Rekonsiliasi Data Berkala.}  
Melakukan audit data secara berkala membantu mengidentifikasi inkonsistensi antar sistem. Proses rekonsiliasi membandingkan data antara sumber untuk mendeteksi dan memperbaiki ketidaksesuaian, seperti transaksi keuangan yang tidak cocok antara sistem akuntansi dan penjualan \cite{laurila1999data}.

\textbf{6. Kebijakan Tata Kelola Data yang Jelas.}  
Menetapkan kebijakan yang mendefinisikan kepemilikan data, tanggung jawab pembaruan, dan prosedur manajemen perubahan memastikan akuntabilitas dan konsistensi dalam pengelolaan data di seluruh organisasi.

Penerapan metode-metode ini meningkatkan keandalan data, mendukung pelaporan yang akurat, dan meningkatkan efisiensi organisasi dengan memastikan semua departemen bekerja menggunakan informasi yang konsisten dan dapat dipercaya.

\subsection{Implikasi dari Data yang Tidak Konsisten}

Data yang tidak konsisten dapat memiliki implikasi serius bagi organisasi, mempengaruhi pengambilan keputusan, efisiensi operasional, kepatuhan, dan kepuasan pelanggan \cite{redman1996impact}.

\textbf{1. Pengambilan Keputusan yang Buruk.}  
Data yang tidak konsisten menghasilkan laporan yang bertentangan dan hasil analisis yang tidak dapat diandalkan. Misalnya, jika angka penjualan berbeda antara departemen keuangan dan penjualan, manajemen dapat membuat keputusan yang salah terkait anggaran, pengadaan inventaris, atau strategi pasar.

\textbf{2. Efisiensi Operasional yang Menurun.}  
Ketika sistem yang berbeda menyimpan data yang tidak konsisten, karyawan dapat menghabiskan banyak waktu untuk merekonsiliasi dan memverifikasi informasi sebelum menyelesaikan tugas, yang memperlambat proses dan meningkatkan biaya operasional \cite{madnick2009overview}.

\textbf{3. Risiko Kepatuhan.}  
Banyak industri memerlukan pelaporan data yang akurat dan konsisten untuk kepatuhan hukum dan regulasi. Inkonsistensi dapat mengakibatkan pelaporan yang salah, denda hukum, dan merusak reputasi organisasi \cite{wang2006data}.

\textbf{4. Ketidakpuasan Pelanggan.}  
Data pelanggan yang tidak konsisten antar sistem (misalnya alamat atau detail kontak yang kedaluwarsa) dapat menyebabkan pengiriman gagal, penagihan yang salah, atau layanan yang tidak dipersonalisasi, menurunkan kepercayaan dan loyalitas pelanggan.

\textbf{5. Peningkatan Biaya.}  
Menyelesaikan inkonsistensi setelah terjadi sering membutuhkan sumber daya yang signifikan, termasuk pembersihan data manual, pembaruan sistem, dan redesain proses, sehingga meningkatkan biaya operasional \cite{pipino2002dataquality}.

\textbf{6. Hilangnya Peluang.}  
Jika data terintegrasi tidak dapat diandalkan, organisasi dapat kehilangan tren pasar, wawasan pelanggan, atau perbaikan operasional yang dapat memberikan keunggulan kompetitif.

Memahami implikasi-implikasi ini menegaskan mengapa menjaga konsistensi data sangat penting bagi organisasi yang ingin mengambil keputusan yang tepat, mempertahankan efisiensi, mematuhi regulasi, dan memberikan layanan berkualitas tinggi kepada pelanggan.


\section{Studi Kasus}

\subsection{Gambaran Umum}

\begin{figure}[h]
	\centering
	\begin{tikzpicture}[
		node distance=1cm and 1cm,
		system/.style={rectangle, draw=black, fill=green!20, rounded corners, text width=3.5cm, minimum height=1cm, align=center, font=\small},
		repo/.style={rectangle, draw=black, fill=blue!20, rounded corners, text width=4.5cm, minimum height=1cm, align=center, font=\small},
		arrow/.style={->, thick}
		]
		
		% Systems
		\node[system] (pos) {\textbf{POS}\\Pembelian Toko Fisik};
		\node[system, right=of pos] (ecom) {\textbf{E-commerce}\\Pembelian Online, Ulasan, Pengiriman};
		\node[system, right=of ecom] (loyalty) {\textbf{Loyalty Programme}\\Keanggotaan, Poin, Hadiah};
		
		% Customer Data Repository
		\node[repo, below=1cm of ecom] (cdr) {\textbf{Customer Data Repository}\\Repositori Terpadu Pelanggan};
		
		% Arrows
		\draw[arrow] (pos) -- (cdr);
		\draw[arrow] (ecom) -- (cdr);
		\draw[arrow] (loyalty) -- (cdr);
		
	\end{tikzpicture}
	\caption{Integrasi Sistem RetailCo ke Customer Data Repository Terpadu}
	\label{fig:retailco-integration}
\end{figure}





\textbf{Kasus: Integrasi Data Pelanggan di RetailCo}

RetailCo adalah perusahaan ritel skala menengah dengan toko fisik di beberapa kota dan platform e-commerce online. Selama bertahun-tahun, data pelanggan dikumpulkan dan disimpan di sistem yang berbeda:

\begin{itemize}
	\item Sistem point-of-sale (POS) menyimpan catatan pembelian pelanggan di toko fisik.
	\item Platform e-commerce mencatat riwayat pembelian online, ulasan pelanggan, dan detail pengiriman.
	\item Program loyalitas pelanggan menyimpan informasi keanggotaan, poin yang dikumpulkan, dan hadiah yang ditukar.
\end{itemize}

Manajemen ingin meluncurkan kampanye pemasaran personalisasi yang menargetkan pelanggan bernilai tinggi di semua kanal. Namun, mereka menemukan adanya inkonsistensi, duplikasi data, dan nilai yang hilang ketika mencoba menggabungkan data dari ketiga sistem tersebut.

Masalah utama yang diidentifikasi:

\begin{enumerate}
	\item Nama pelanggan dieja berbeda di setiap sistem (misalnya “John Smith” vs. “Jon Smith”).
	\item Platform e-commerce mencatat tanggal lahir, tetapi sistem POS tidak.
	\item Beberapa alamat email di database program loyalitas sudah kedaluwarsa atau hilang.
	\item ID pelanggan bersifat spesifik sistem tanpa adanya pengidentifikasi yang terintegrasi.
\end{enumerate}

\subsection{Analisis}

Situasi RetailCo menyoroti tantangan umum dalam pembersihan, persiapan, dan integrasi data:

\begin{enumerate}
	\item \textbf{Kebutuhan Pembersihan Data.}
	\begin{itemize}
		\item Rekaman duplikat untuk pelanggan yang sama dengan ejaan nama berbeda memerlukan deduplikasi dan standarisasi.
	\end{itemize}
	
	\item \textbf{Kesenjangan Persiapan Data.}
	\begin{itemize}
		\item Tidak adanya data tanggal lahir di sistem POS membatasi segmentasi demografis.
		\item Solusi yang mungkin termasuk mengumpulkan data ini di toko fisik atau mengestimasi demografi berdasarkan pola pembelian.
	\end{itemize}
	
	\item \textbf{Tantangan Integrasi.}
	\begin{itemize}
		\item Perbedaan sistem ID pelanggan mencegah penggabungan rekaman secara langsung.
		\item Implementasi sistem master data management (MDM) diperlukan untuk menyatukan pengidentifikasi pelanggan.
	\end{itemize}
	
	\item \textbf{Masalah Konsistensi.}
	\begin{itemize}
		\item Alamat email yang kedaluwarsa atau inkonsisten mengurangi efektivitas kampanye pemasaran.
		\item Proses validasi dan pembaruan data diperlukan untuk menjaga relevansi data.
	\end{itemize}
	
	\item \textbf{Dampak Bisnis.}
	\begin{itemize}
		\item Penargetan pemasaran menjadi tidak efektif.
		\item Ketidakpuasan pelanggan akibat penawaran yang tidak relevan.
		\item Hilangnya peluang pendapatan dari promosi lintas kanal.
	\end{itemize}
\end{enumerate}

Mengatasi tantangan ini meliputi:

\begin{enumerate}
	\item Menerapkan teknik pembersihan data seperti standarisasi format nama dan deduplikasi rekaman.
	\item Mengintegrasikan data menggunakan proses ETL yang dikombinasikan dengan sistem MDM untuk menyatukan ID pelanggan.
	\item Menetapkan kebijakan tata kelola data untuk menjaga kualitas data di semua sistem.
\end{enumerate}

Kasus ini menggambarkan pentingnya pembersihan dan persiapan data sebagai prasyarat untuk integrasi data yang berhasil dan inisiatif bisnis strategis.

\section{Aktivitas Praktik: Mengevaluasi dan Mempersiapkan Data Tanpa Coding}

\textbf{Tujuan.}  
Berlatih mengevaluasi data, membersihkan, mempersiapkan, dan merencanakan integrasi menggunakan dataset nyata dengan tugas terarah dan solusi yang diharapkan.

\textbf{Dataset.}

Gunakan \textbf{dataset Online Retail} berikut:

\begin{itemize}
		\item \url{https://archive.ics.uci.edu/dataset/352/online+retail}.
	\item \url{https://archive.ics.uci.edu/ml/machine-learning-databases/00352/Online%20Retail.xlsx} 
	\item Buka file Excel dan tinjau data pada sheet yang tersedia.
\end{itemize}

\begin{table}[h]
	\centering
	\scriptsize
	\renewcommand{\arraystretch}{1.2}
			\caption{Cuplikan Dataset Online Retail untuk Praktik Pembersihan dan Persiapan Data}
	\begin{tabular}{|p{0.08\textwidth}|p{0.08\textwidth}|p{0.15\textwidth}|p{0.06\textwidth}|p{0.1\textwidth}|p{0.08\textwidth}|p{0.08\textwidth}|p{0.1\textwidth}|}
		\hline
		\textbf{InvoiceNo} & \textbf{StockCode} & \textbf{Description} & \textbf{Quantity} & \textbf{InvoiceDate} & \textbf{UnitPrice} & \textbf{CustomerID} & \textbf{Country} \\
		\hline
		536365 & 85123A & WHITE HANGING HEART T-LIGHT HOLDER & 6 & 01/12/2010 08:26 & 2.55 & 17850 & United Kingdom \\
		\hline
		536365 & 71053 & WHITE METAL LANTERN & 6 & 01/12/2010 08:26 & 3.39 & 17850 & United Kingdom \\
		\hline
		\multicolumn{8}{c}{\ldots} \\
		\hline
	\end{tabular}

	\label{tab:online-retail-snippet}
\end{table}


\textbf{Instruksi.}

Lakukan langkah-langkah berikut secara individu:

\begin{enumerate}
	\item \textbf{Identifikasi Masalah Kualitas Data.}
	
	Tinjau dataset dan catat:
	
	\begin{enumerate}
		\item \textbf{Nilai Hilang.} Filter kolom CustomerID untuk menemukan entri yang kosong.
		\item \textbf{Ketidakkonsistenan Deskripsi.} Periksa deskripsi produk yang ditulis seluruhnya dengan huruf kapital, huruf kecil, atau campuran.
		\item \textbf{Jumlah Negatif.} Urutkan Quantity untuk mengidentifikasi nilai negatif yang menunjukkan pengembalian barang.
	\end{enumerate}
	
	\textbf{Solusi yang Diharapkan:}
	
	\begin{itemize}
		\item Untuk CustomerID yang hilang: usulkan untuk menghapus baris tersebut jika analisis memerlukan segmentasi pelanggan, atau perlakukan sebagai pelanggan anonim.
		\item Untuk deskripsi yang tidak konsisten: standarisasi menjadi format title case untuk konsistensi.
		\item Untuk jumlah negatif: klasifikasikan sebagai retur dan buat kolom penanda untuk item yang dikembalikan.
	\end{itemize}
	
	\item \textbf{Mempersiapkan Data untuk Analisis.}
	
	\begin{enumerate}
		\item Buat kolom baru untuk mengkategorikan UnitPrice:
		\begin{itemize}
			\item UnitPrice < £1: “Harga Rendah”
			\item UnitPrice £1–£10: “Harga Menengah”
			\item UnitPrice > £10: “Harga Tinggi”
		\end{itemize}
		\item Ubah format InvoiceDate untuk mengekstrak hanya bulan guna analisis tren penjualan bulanan.
	\end{enumerate}
	
	\textbf{Solusi yang Diharapkan:}
	
	\begin{itemize}
		\item Gunakan rumus IF di Excel untuk mengkategorikan nilai UnitPrice.
		\item Gunakan fungsi MONTH atau TEXT di Excel untuk mengekstrak bulan dari InvoiceDate.
	\end{itemize}
	
	\item \textbf{Perencanaan Integrasi.}
	
	\begin{enumerate}
		\item Bayangkan dataset kedua yang berisi data demografi pelanggan dengan CustomerID, AgeGroup, dan LoyaltyStatus. 
		\item Lakukan dengan membuat Sheet baru dengan kolom-kolom tersebut. Isi CustomerID dengan beberapa CustomerID yang ada. Isi AgeGroup dan LoyaltyStatus dengan nilai yang ditentukan oleh Anda sendiri.
		\item Usulkan rencana integrasi untuk menggabungkan dataset tersebut dengan dataset Online Retail untuk analisis segmentasi pelanggan.
	\end{enumerate}
	
	\textbf{Solusi yang Diharapkan:}
	
	\begin{itemize}
		\item Lakukan left join menggunakan CustomerID sebagai kunci untuk menambahkan informasi demografi ke setiap catatan transaksi.
	\end{itemize}
	
	\item \textbf{Refleksi.}
	
	Tulis jawaban singkat untuk:
	
	\begin{enumerate}
		\item Mengapa konsistensi data penting dalam menganalisis data penjualan dan perilaku pelanggan?
		\item Tantangan organisasi atau teknis apa yang dapat muncul saat mengintegrasikan data transaksi dan demografi?
	\end{enumerate}
	
	\textbf{Solusi yang Diharapkan:}
	
	\begin{itemize}
		\item Konsistensi memastikan akurasi analisis dan wawasan yang dapat diandalkan; inkonsistensi menyebabkan penargetan atau prakiraan yang salah.
		\item Tantangan meliputi kepemilikan data yang berbeda antar departemen, format data yang tidak kompatibel, atau pembatasan privasi dan akses.
	\end{itemize}
	
	
\end{enumerate}

\textbf{Hasil yang Diharapkan.}  
Latihan ini bertujuan membangun kepercayaan diri praktis dalam mengevaluasi dan mempersiapkan data bisnis nyata untuk pengambilan keputusan, menggunakan langkah-langkah terstruktur tanpa coding.

\section{Ringkasan}

Bab ini telah memperkenalkan konsep pembersihan dan persiapan data, menekankan pentingnya memastikan data bisnis akurat, lengkap, dan siap untuk dianalisis. Dibahas teknik utama pembersihan data seperti menangani nilai hilang, mengelola outlier, dan normalisasi data, serta proses persiapan yang lebih luas termasuk integrasi dan menjaga konsistensi antar sistem. Langkah-langkah dasar ini penting untuk mengubah data mentah menjadi informasi yang andal yang mendukung pengambilan keputusan dan perencanaan strategis yang efektif.

Bab ini juga membahas tantangan praktis dalam menerapkan inisiatif pembersihan dan integrasi data, menyoroti masalah organisasi seperti kepemilikan dan tata kelola data, serta keterbatasan teknis seperti kompatibilitas sistem dan kendala skalabilitas. Melalui aktivitas praktik menggunakan dataset Online Retail, mahasiswa mendapatkan pengalaman langsung dalam mengevaluasi kualitas data, merencanakan langkah pembersihan, dan mengintegrasikan dataset tanpa coding. Memahami konsep ini membekali calon manajer untuk memimpin proyek berbasis data dengan percaya diri dan berkolaborasi secara efektif dengan tim teknis untuk memaksimalkan nilai bisnis dari aset data.

